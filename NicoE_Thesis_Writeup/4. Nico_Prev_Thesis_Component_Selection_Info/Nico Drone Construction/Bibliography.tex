% \chapter*{Bibliography}
\addcontentsline{toc}{chapter}{Bibliography}

\begin{thebibliography}{99}

    \bibitem{compSLAM} Khattak, Shehryar, et al. "Complementary multi–modal sensor fusion for resilient robot pose estimation in subterranean environments." 2020 International Conference on Unmanned Aircraft Systems (ICUAS). IEEE, 2020.



	\bibitem{RMFOwl} De Petris, Paolo, et al. ``Rmf-owl: A collision-tolerant flying robot for autonomous subterranean exploration.'' 2022 International Conference on Unmanned Aircraft Systems (ICUAS). IEEE, 2022.
	
	\bibitem{CoSTAR}Otsu, Kyohei, et al. ``Supervised autonomy for communication-degraded subterranean exploration by a robot team.'' 2020 IEEE Aerospace Conference. IEEE, 2020.
	
	\bibitem{LAMP} Ebadi, Kamak, et al. ``LAMP: Large-scale autonomous mapping and positioning for exploration of perceptually-degraded subterranean environments.'' 2020 IEEE International Conference on Robotics and Automation (ICRA). IEEE, 2020.
	
	\bibitem{GraphBased} Dang, Tung, et al. ``Graph-based path planning for autonomous robotic exploration in subterranean environments.'' 2019 IEEE/RSJ International Conference on Intelligent Robots and Systems (IROS). IEEE, 2019.
	
	\bibitem{ExploreLocally} Dang, Tung, et al. ``Explore locally, plan globally: A path planning framework for autonomous robotic exploration in subterranean environments.'' 2019 19th International Conference on Advanced Robotics (ICAR). IEEE, 2019.
	
	\bibitem{GazeboCave} Koval, Anton, et al. ``A subterranean virtual cave world for gazebo based on the DARPA SubT challenge.'' arXiv preprint arXiv:2004.08452 (2020).
	
	\bibitem{HeterogeneousSystem} Biggie, Harel, and Steve McGuire. ``Heterogeneous Ground-Air Autonomous Vehicle Networking in Austere Environments: Practical Implementation of a Mesh Network in the DARPA Subterranean Challenge.'' 2022 18th International Conference on Distributed Computing in Sensor Systems (DCOSS). IEEE, 2022.
	
	\bibitem{DarpaSynopsis} Orekhov, V., and T. Chung. ``The DARPA subterranean challenge: A synopsis of the circuits stage.'' Field Robotics 2.1 (2022): 735-747.

	\bibitem{DarpaMultiRobot}Rouček, Tomáš, et al. ``Darpa subterranean challenge: Multi-robotic exploration of underground environments.'' Modelling and Simulation for Autonomous Systems: 6th International Conference, MESAS 2019, Palermo, Italy, October 29–31, 2019, Revised Selected Papers 6. Springer International Publishing, 2020.
	
	\bibitem{Cerberus} Tranzatto, Marco, et al. ``CERBERUS in the DARPA Subterranean Challenge.'' Science Robotics 7.66 (2022): eabp9742.
	
	\bibitem{Marble} Riley, Danny G., and Eric W. Frew. ``Fielded Human-Robot Interaction for a Heterogeneous Team in the DARPA Subterranean Challenge.'' ACM Transactions on Human-Robot Interaction (2023).
	
	\bibitem{STEP}Dixit, Anushri, et al. ``STEP: Stochastic Traversability Evaluation and Planning for Risk-Aware Off-road Navigation; Results from the DARPA Subterranean Challenge.'' arXiv preprint arXiv:2303.01614 (2023).
	
	\bibitem{Nebula} ``NeBula: Quest for robotic autonomy in challenging environments; an overview of TEAM CoSTAR's Solution at phase I and II of DARPA Subterranean challenge''
	
	\bibitem{Norlab} Roucek, Tomáš, et al. ``System for multi-robotic exploration of underground environments CTU-CRAS-NORLAB in the DARPA Subterranean Challenge.'' arXiv preprint arXiv:2110.05911 (2021).
	
	\bibitem{MultiAgent} Ohradzansky, Michael T., et al. ``Multi-agent autonomy: Advancements and challenges in subterranean exploration.'' arXiv preprint arXiv:2110.04390 (2021).
	
	\bibitem{Cerberus2} Tranzatto, Marco, et al. ``Cerberus: Autonomous legged and aerial robotic exploration in the tunnel and urban circuits of the darpa subterranean challenge.'' arXiv preprint arXiv:2201.07067 (2022).
	
	\bibitem{Supervised} Biggie, Harel, et al. ``Flexible supervised autonomy for exploration in subterranean environments.'' arXiv preprint arXiv:2301.00771 (2023).

	\bibitem{miningRobotCalibration} P. Trybała, J. Szrek, F. Remondino, J. Wodecki, and R. Zimroz, ``CALIBRATION OF A MULTI-SENSOR WHEELED ROBOT FOR THE 3D MAPPING OF UNDERGROUND MINING TUNNELS.'' The International Archives of the Photogrammetry, Remote Sensing and Spatial Information Sciences, vol. XLVIII-2-W2-2022, pp. 135–142, Dec. 2022, doi: 10.5194/isprs-archives-XLVIII-2-W2-2022-135-2022.

	\bibitem{emesent360LiDAR} J. Marteau-Reay, ``Emesent launches Handheld 360 Image Kit for Hovermap.'' Emesent. Accessed: Feb. 12, 2024. [Online]. Available: https://emesent.com/2024/01/24/emesent-launched-handheld-360-image-kit-for-hovermap/

	\bibitem{scannerReviews} Soilán, Sánchez-Rodríguez, Río-Barral, Perez-Collazo, Arias and Riveiro (2019). Review of Laser Scanning Technologies and Their Applications for Road and Railway Infrastructure Monitoring. Infrastructures, 4(4), p.58. doi:https://doi.org/10.3390/infrastructures4040058.
	
\end{thebibliography}


